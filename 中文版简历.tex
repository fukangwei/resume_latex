%!TEX TS-program = xelatex
%!TEX encoding = UTF-8 Unicode
\documentclass[11pt, a4paper]{moderncv}
\usepackage[noindent, UTF8]{ctex}
\usepackage{fontspec}
% moderncv themes. optional argument are 'blue' (default), 'orange', 'red', 
% 'green', 'grey' and 'roman' (for roman fonts, instead of sans serif fonts)
\moderncvtheme[blue]{classic}
\usepackage{xunicode, xltxtra}
\XeTeXlinebreaklocale "zh"
\widowpenalty=10000

% adjust the page margins
\usepackage[scale=0.8]{geometry}
\recomputelengths  % required when changes are made to page layout lengths

% 个人信息
\firstname{付}
\familyname{康为}
\title{个人简历}
\address{江苏省徐州市铜山区大彭镇}{221100 徐州}
\mobile{+86~19895945153}
\email{fukangwei\_lite@163.com}
\homepage{https://fukangwei.github.io}                 
\extrainfo{求职意向:C/C++软件工程师,Python软件工程师,图像处理工程师,深度学习工程师}
\photo[82pt]{fukangwei.jpg}

\begin{document}
	\maketitle
	
	\section{教育背景}
	\cventry{2012.9 -- 2016.6}{工学学士}{南通大学电子信息学院}{集成电路与集成系统}{}{}
	\cventry{2016.9 -- 2019.9}{工学硕士}{南通大学电子信息学院}{信息与通信工程}{第一学位}{}
	\cventry{2018.4 -- 2020.4}{工学修士}{日本德岛大学}{知能情报专业}{第二学位}{}

	\section{社区}
	\cventry{笔记+博客}{\url{https://fukangwei.github.io}}{}{}{}{}
	\cventry{}{\url{https://fukangwei.gitee.io}}{}{}{}{}
	\cventry{GitHub}{\url{http://github.com/fukangwei}}{}{}{}{}
	
	\section{实习经历}
	\cventry{2019.5 -- 至今}
	{苏州博世汽车部件有限公司}
	{}{}{}{
		\begin{itemize}
			\item 主要职位:深度学习实习生、单元测试实习生、Android开发实习生。
			\begin{enumerate}
				\item 深度学习实习生:在Ubuntu上搭建NVIDIA Drive PX2系统的开发环境,该系统是一种自动驾驶开发平台。
				\item 单元测试实习生:参与CC-EAV部门的代码单元测试,使用cppunit进行测试覆盖。
			\end{enumerate}
			\item 主要技术:Ubuntu、深度学习、cppunit、CAN总线。
	\end{itemize}}{}	 
	
	\section{项目经历}
	\cventry{2018.9 -- 2019.1}
	{基于MobileNet v2和U-Net的自然场景文字检测系统}
	{}{}{}{
		\begin{itemize}
			\item 项目简介:针对自然场景图片进行语义分割,以识别文字像素区域与非文字像素区域。
			\item 主要职责:对数据集进行数据增强,设计出MobileNet v2 + U-Net神经网络模型。
			\item 主要技术:MobileNet v2,U-Net,数据增强,OpenCV,图像语义分割。
			\item 项目结果:极大地减小了神经网络模型的复杂度,识别率良好。
		\end{itemize}}{}	  

	\cventry{2018.4 -- 2018.8}
	{基于SSD神经网络的危险物品识别系统}
	{}{}{}{
		\begin{itemize}
			\item 项目简介:基于SSD目标检测与ResNet神经网络模型,识别刀具等危险物品。
			\item 主要职责:修改SSD目标检测的神经网络模型,将VGG替换为ResNet。
			\item 主要技术:SSD目标检测模型,迁移学习,ResNet网络,Keras深度学习框架
			\item 项目结果:可以检测出图片中的危险物品,并且检测的正确率高于原始的SSD模型。
	\end{itemize}}{}

	\cventry{2017.6 -- 2017.9}
	{基于深度学习的人脸表情识别系统}
	{}{}{}{
		\begin{itemize}
			\item 项目简介:使用Pytorch搭建神经网络模型,对人脸的表情进行分类。
			\item 主要职责:对原始数据集进行数据增强,以防止过拟合;搭建卷积神经网络模型。
			\item 主要技术:Pytorch深度学习框架,卷积神经网络,数据增强,混淆矩阵
			\item 项目结果:使用Fer2013数据集进行训练,分类准确率可以达到65\%。
	\end{itemize}}{}

	\cventry{2016.1 -- 2016.9}
	{基于IPv6的现代农业物联网技术及应用}
	{}{}{}{
		\begin{itemize}
			\item 项目简介:搭建基于6LowPAN的无线传感器网络,使用IPv6协议传输传感器数据。
			\item 主要职责:6LowPAN协议栈移植,应用层软件编写,使用Qt实现UI界面。
			\item 主要技术:6LowPAN无线传感网络协议栈,Qt,Contiki嵌入式操作系统
			\item 项目结果:实现了6LowPAN无线传感器网络,传感器通过IPv6协议传输数据。
	\end{itemize}}{}

	\cventry{2015.3 -- 2015.8}
	{基于北斗的巡航船系统设计}
	{}{}{}{
		\begin{itemize}
			\item 项目简介:基于北斗定位,使巡航船实现自主循迹,并向服务器发送位置信息。
			\item 主要职责:编写嵌入式软件,编写北斗模块以及GPRS模块的设备驱动。
			\item 主要技术:STM32,北斗模块,GPRS模块,PID算法
			\item 项目结果:可以实现巡航船的循迹,以及实时跟踪船的位置。
	\end{itemize}}{}

	\section{语言技能}
	\cvline{汉语}{母语。}
	\cvline{英语}{\textbf{CET-6},经常阅读英文论文,在国外可以用英语和老师交流。}
	\cvline{日语}{在日本留学一年,可以进行日常交流。}

	\section{专业技能}
	\cvline{编程语言}{C > C++ > Python > Shell}
	\cvline{工具}{Markdown,LaTeX,Github,Hexo,Office}
	\cvline{操作系统}{GNU/Linux(Ubuntu, CentOS),Windows}
	\cvline{深度学习框架}{Pytorch > Keras > TensorFlow}
	\cvline{机器学习框架}{Sklearn}
	\cvline{视觉处理工具库}{OpenCV}
	\cvline{神经网络模型}{CNN,FCN,FPN,U-Net,MobileNet,MobileNet V2,VGG,ResNet}
	\cvline{嵌入式系统}{80C51,STM32,S3C2440,CC2530}
	\cvline{数据库}{SQLite,MySQL}
	\cvline{应用程序框架}{Qt}

	\section{获得奖励}
	\cventry{2015.5}
	{2015年江苏省大学生计算机设计大赛软件服务外包类本科组优胜奖}{}{}{}{}
	\cventry{2015.6}
	{第十四届“挑战杯”江苏省大学生课外学术科技作品竞赛三等奖}{}{}{}{}
	\cventry{2015.8}
	{全国大学生物联网设计竞赛华东赛区二等奖}{}{}{}{}
	\cventry{2016.5}
	{“中国动力谷杯”第十一届全国大学生交通科技大赛二等奖}{}{}{}{}
	\cventry{2016.7}
	{“华为杯”第十一届中国研究生电子设计大赛华东分赛区团队二等奖}{}{}{}{}
	\cventry{2017.7}
	{“华为杯”第十二届中国研究生电子设计大赛华东分赛区团队三等奖 }{}{}{}{}

	\section{个人兴趣}
	\cvline{爱好}{钻研技术,整理笔记,写博客,修改BUG}
	\cvline{互联网}{\href{http://github.com/fukangwei}{GitHub}}
	\cvline{其他}{摄影,看书,做饭}
\end{document}