%!TEX TS-program = xelatex
%!TEX encoding = UTF-8 Unicode
\documentclass[11pt, a4paper]{moderncv}
\usepackage[noindent, UTF8]{ctex}
\usepackage{fontspec}
% moderncv themes. optional argument are 'blue' (default), 'orange', 'red', 
% 'green', 'grey' and 'roman' (for roman fonts, instead of sans serif fonts)
\moderncvtheme[blue]{classic}
\usepackage{xunicode, xltxtra}
\XeTeXlinebreaklocale "zh"
\widowpenalty=10000

% adjust the page margins
\usepackage[scale=0.9]{geometry}
\recomputelengths  % required when changes are made to page layout lengths

% 个人信息
\firstname{Fu}
\familyname{Kangwei}
\title{Personal Resume}
\address{No.9\,Seyuan\,Road,\,Chongchuan\,District,\,Nantong,\,Jiangsu}{226019 Nantong}
\mobile{+86~19895945153}
\email{fukangwei\_lite@163.com}
\homepage{https://fukangwei.github.io}                 
\extrainfo{JOB OBJECTIVE: C/C++/Python Engineer\\Image Processing Engineer\\Deep Learning Engineer}
\photo[82pt]{fukangwei.jpg}

\begin{document}
	\maketitle
	
	\section{Education Background}
	\cventry{2012.9 -- 2016.6}{Bachelor of Engineering}{Nantong University, School of Electronics and Information Engineering}{Integrated Circuits and Integrated Systems}{}{}
	\cventry{2016.9 -- Now}{Master of Engineering}{Nantong University, School of Electronics and Information Engineering}{Information and Communication Engineering}{\textbf{First Degree}}{}
	\cventry{2018.4 -- Now}{Master of Engineering}{Tokushima University}{Intelligent Information}{\textbf{Second Degree}}{}

	\section{Community}
	\cventry{Note + Blog}{\url{https://fukangwei.github.io}}{}{}{}{}
	\cventry{GitHub}{\url{http://github.com/fukangwei}}{}{}{}{}
	
	\section{Project Experience}
	\cventry{2018.9 -- 2019.1}
	{Natural Scene Text Detection System Based on MobileNet v2 and U-Net}
	{}{}{}{
		\begin{itemize}
			\item Project Description: Do semantic segmentation for natural scene images, and distinguish between text areas and non-text areas.
			\item Major Duty: Augment the dataset; Design the neural network architecture inspired by MobileNet V2 and U-Net.
			\item Technology Used: MobileNet v2, U-Net, Data Augmentation, OpenCV, Semantic Segmentation
			\item Project Results: The complexity of neural network is reduced greatly, and the accuracy is good.
		\end{itemize}}{}	  

	\cventry{2018.4 -- 2018.8}
	{Dangerous Goods Detection System Based on SSD Neural Network}
	{}{}{}{
		\begin{itemize}
			\item Project Description: This system which is based on SSD and ResNet can detect the dangerous goods such as knives.
			\item Major Duty: Modify the neural network architecture of SSD target detection (Replace VGG with ResNet).
			\item Technology Used: Single Shot MultiBox Detector, Transfer Learning, ResNet, Keras
			\item Project Results: Dangerous goods in the image can be detected, and the accuracy is higher than the original SSD architecture.
	\end{itemize}}{}

	\cventry{2017.6 -- 2017.9}
	{Facial Expression Recognition System Based on Deep Learning}
	{}{}{}{
		\begin{itemize}
			\item Project Description: Use Pytorch to build a neural network model used for classifying facial expressions.
			\item Major Duty: Augment the dataset for solving overfiting, Build convolutional neural network model
			\item Technology Used: Pytorch, CNN, Data Augmentation, Confusion Matrix
			\item Project Results: Using the Fer2013 dataset for training, and the classification accuracy can reach to 65\%.
	\end{itemize}}{}

	\cventry{2016.1 -- 2016.9}
	{The IOT Application for Agriculture Based on IPv6}
	{}{}{}{
		\begin{itemize}
			\item Project Description: Build the wireless sensor network based on 6LowPAN, and transmit sensor data by the IPv6 protocol.
			\item Major Duty: Transplant 6LowPAN code, Write embedding application, Write computer application with QT 
			\item Technology Used: 6LowPAN, QT, Contiki
			\item Project Results: Build the 6LowPAN wireless sensor network, and the sensors can transmit data by the IPv6 protocol.
	\end{itemize}}{}

	\cventry{2015.3 -- 2015.8}
	{The Design of Autonomous Tracking Ship Based on Beidou Navigation Satellite System}
	{}{}{}{
		\begin{itemize}
			\item Project Description: The ship realizes the autonomous tracking with Beidou Navigation Satellite System, and sends the location information to the server.
			\item Major Duty: Write Embedding Application, Write Device Drivers for Beidou and GPRS Module
			\item Technology Used: STM32, BDS(Beidou Navigation Satellite System), GPRS, PID, Kalman Filter
			\item Project Results: The ship owns the function of autonomous tracking, and we can track the position of the ship in real time. 
	\end{itemize}}{}

	\section{Language Skill}
	\cvline{Chinese}{mother tongue}
	\cvline{English}{\textbf{CET-6} level, Always read English papers, and communicate with teachers using English abroad}
	\cvline{Japanese}{Study in Japan for one year, and be able to conduct daily communication}

	\section{Professional Skill}
	\cvline{Programing Language}{Python > C++ > C > Java > Node.js > HTML > Shell}
	\cvline{Tool}{Markdown, LaTeX, Github, Hexo, Office}
	\cvline{Operation\,System}{GNU/Linux(Ubuntu, CentOS), Windows}
	\cvline{Deep\,Learning}{Pytorch > Keras > TensorFlow}
	\cvline{Machine\,Learning}{Sklearn}
	\cvline{Computer\,Vision Library}{OpenCV}
	\cvline{Neural\,Network\\model}{CNN, FCN, FPN, U-Net, MobileNet, MobileNet V2, VGG, ResNet}
	\cvline{Embedding\,System}{80C51, STM32, S3C2440, CC2530, Raspberry Pi}
	\cvline{Database}{SQLite}
	\cvline{Application Framework}{QT, Android}

	\section{Rewards}
	\cventry{2015.5}
	{Undergraduate Group Award for Software Service in College Student Computer Design Competition in Jiangsu}{}{}{}{}
	\cventry{2015.6}
	{The Third Prize of the 14th "Challenge Cup" for College Students' Extracurricular Academic and Scientific Works Competition in Jiangsu}{}{}{}{}
	\cventry{2015.8}
	{The Second Prize of National College Students IOT Design Competition in East China Division}{}{}{}{}
	\cventry{2016.5}
	{The Second Prize of the 11th "China Power Valley Cup" for National College Students Transportation Science and Technology Competition}{}{}{}{}
	\cventry{2016.7}
	{The Team Second Prize of 11th "Huawei Cup" for China Graduate Electronic Design Competition in East China Division}{}{}{}{}
	\cventry{2017.7}
	{The Team Third Prize of 12th "Huawei Cup" for China Graduate Electronic Design Competition in East China Division}{}{}{}{}

	\section{Personal Interest}
	\cvline{Hobby}{Research Technology, Collate Notes, Write Blogs, Fix Bugs}
	\cvline{Internet}{\href{http://github.com/fukangwei}{GitHub}}
	\cvline{Others}{Photography, Read Books}
\end{document}